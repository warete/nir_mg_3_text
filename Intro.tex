\intro
В этом файле должен находиться текст раздела <<Введение>>. 


Во <<Введении>>  требуется обосновать актуальность выбранной темы, которая определяется значимостью ее теоретического и практического решения. Формулируются цель и задачи практики, определяются объект и предмет исследования, методология исследования. Целесообразно также охарактеризовать степень разработанности темы в отечественной и зарубежной литературе, изложить структуру работы. Этот раздел должен включать в себя краткое содержание осовных разделов отчета. Введение не должно составлять более 8--10\% от общего объема отчета (3--5 страниц). Так как <<Введение>> содержит в себе оприсание постановки задачи и освещает то, как тема разработана ранее другими авторами, то этот раздел предполагает наличие ссылок на другие источники. Приведем пример того, как ссылки должны быть оформлены~\cite{CitekeyArticle, Boreskov2010}.

