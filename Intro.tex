\intro
В последние несколько лет в различных сферах жизни человека очень часто используются алгоритмы машинного обучения и нейросети. Они помогают решать совершенно разные задачи - от подбора персональных рекомендаций для просмотра фильмов, до помощи в диагностировании различных заболеваний на ранней стадии. В процессе разработки программного обеспечения с подобным функционалом происходит очень много итераций по настройке модели и подбору параметров для нее. Эти рутинные операции в основном производятся с помощью замены обучающей выборки, изменения значений параметров в коде и перезапуска программы. На некоторых этапах разработки может потребоваться показать свои результаты другому человеку, возможно далекому от деталей реализации получившейся нейросетевой модели и появляется необходимость в простом универсальном пользовательском интерфейсе, который можно было бы к ней подключить и использовать.

В данной работе рассматриваются области применения нейросетей, популярные библиотеки с их реализацией и существующие в данный момент решения на рынке для подключения к ним графического интерфейса. 

Одной из целей работы является проектирование и разработка библиотеки с реализацией пользовательского графического интерфейса для задач машинного обучения и нейросетей на языке Python. Так же необходимо оформить получившуюся программу в pip-пакет и опубликовать его в сервисе PyPI.