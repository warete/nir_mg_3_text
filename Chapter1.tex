\chapter{Методы и области применения машинного обучения и нейросетей}

\section{Обзор литературы}
Статья Полякова М.В., Хоперскова А.В. «Математическое моделирование пространственного распределения радиационного поля в биоткани: определение яркостной температуры для диагностики», опубликованная в Вестнике Волгоградского государственного университета, посвящена проведению имитационных экспериментов по моделированию динамики температурных и радиационных полей в биотканях молочной железы. В работе вместо традиционно используемых моделей с однородными параметрами используются вычислительные модели максимально приближенные к реалистичной геометрической структуре тканей с неоднородными характеристиками~\cite{polyakovKhoperskov}.


В статье Веснина С.Г., Седанкина М.К. «Миниатюрные антенны-\\аппликаторы для микроволновых радиотермометров медицинского назначения», опубликованная в журнале «Биомедицинская радиоэлектроника», описывается анализ миниатюрных антенн-аппликаторов, предназначенных для измерения собственного излучения тканей человека с помощью микроволновых радиотермометров. Приведены простые аппроксимационные формулы для распределения температуры в молочной железе при наличиии злокачественной опухоли~\cite{vesninMinAntenn}.


В работе Van Ongeval Ch. «Digital mammography for screening and diagnosis of breast cancer: an overview» обсуждается цифровая телемаммография как новая техника для диагностирования заболеваний молочных желез. Также в данной работе детально рассматривается проблема практической реализации различных систем для визуализации телемаммографической диагностики, высокой стоимости обследования и высокой квалификации специалистов-радиологов~\cite{vanOngeval}.


Работа Nisreen I. Yassin, Shaimaa Omran, Enas M. F. El Houby, Enas M. F. El Houby, Hemat Allam «Machine Learning Techniques for Breast Cancer Computer Aided Diagnosis Using Different Image Modalities: A Systematic Review» посвящена опыту медиков в диагностики и обнаружении рака молочной железы с использованием алгоритмов машинного обучения на основе визуализированных данных обследования пациентов. Целью работы является исследование современного уровня техники в отношении систем компьютерной диагностики и обнаружения рака молочной железы~\cite{nisrenml}.


В статье Левшинского В., Полякова М., Лосева А., Хоперскова А. \\ «Verification and Validation of Computer Models for Diagnosing Breast Cancer Based on Machine Learning for Medical Data Analysis» рассмотрен поход проверки результатов моделирования физических процессов в биотканях с использованием глубокого анализа и машинного обучения. При обучении моделей используются данные измерений температуры пациентов согласно методу радиотермометрии. Так же в работе выделяются на основе набора данных для обучения новые признаки, похожие на те, которые используют медики при обследовании пациентов~\cite{lev-polyakov-losev-hoperskov}.


В работе Рамсундара Б., Истмана П., Уолтерса П., Панде В. «Глубокое обучение в биологии и медицине» обсуждается применение глубокого обучения в популярных направлениях современных исследований, а особенно в биологии и медицине. Работа содержит описание архитектуры алгоритмов в машинном обучении для применения в задачах данных сферах, а так же некоторые практические примеры по использованию~\cite{ramsundar-deep-learning}.


Статья Jian Ma, Pengchao Shang, Chen Lu, Safa Meraghni «A portable breast cancer detection system based on smartphone with infrared camera» посвящена разработке системы обнаружения рака молочной железы с использованием смартфона с инфракрасной камерой. Для обследования использовался метод инфракрасной термографии и алгоритм классификации k-ближайших соседей. Авторам удалось достигнуть точности определения наличия заболевания больше 98\%~\cite{mobile-breast-cancer-detection}.


В части работ рассмотрен метод микроволновой радиотермометрии и его применение при обследовании рака молочных желез. Так же в некоторых работах рассмотрены способы применения машинного обучения для диагностирования различных заболеваний, в том числе онкологических.

\section{Области применения}

Использование алгоритмов машинного обучения и нейросетей позволяет решать задачи в различных сферах деятельности человека, таких как недвижимость, сельское хозяйство, экономика, а так же медицина. По данным агенства Frost \& Sullivan спрос на разработки, в которых используется машинное обучение в медицине, увеличивается с каждым годом примерно на 40\%~\cite{habrbigdatamedicine}. Такие разработки могут использоваться как для диагностики заболеваний, так  и для биохимических исследований.


Методы машинного обучения активно применяются при медицинском сканировании различных типов, таких как УЗИ или компьютерная томография. Благодаря алгоритмам распознавания образов на изображениях есть возможность анализировать результаты таких исследований и указывать на проблемные участки. Также возможно определение диагноза пациента по различным его параметрам и результатам исследования. Но программное обеспечение, использующее данные алгоритмы пока не может заменить полностью работу медиков и используется в основном при первичных исследованиях в качестве экспертных систем.



При компьютерном моделировании алгоритмы машинного обучения могут использоваться для валидации получившихся данных, или прогнозирования течения каких-либо физических процессов.


\section{Популярные библиотеки с реализацией нейронных сетей}

На текущий момент существует множество готовых реализаций нейросетей и алгоритмов машинного обучения, что не имеет смысла делать то же самое с нуля, если задача не имеет каких-то особенностей, делающих невозможным использование готовых библиотек. Каждая из библиотек, рассматриваемых в работе, хороша в своей области, успешно используется в решении задач и проверена временем. Рассмотрим некоторые из популярных библиотек для языка программирования Python по данным рейтинга на GitHub (рисунок~\ref{fig:github_stars})

\imgh{0.9}{github_stars}{Популярные пакеты Python для машинного обучения по данным рейтинга на GitHub}


\subsection{TensorFlow}
Самой популярной и масштабной по применению является библиотека TensorFlow, используемая для глубокого машинного обучения~\cite{gudfellow}. Библиотека разрабатывается в тесном сотрудничестве с компанией Google и применяется в большинстве их проектов где используется машинное обучение. Библиотека использует систему многоуровневых узлов, которая позволяет вам быстро настраивать, обучать и развертывать искусственные нейронные сети с большими наборами данных.


Библиотека хорошо подходит для широкого семейства техник машинного обучения, а не только для глубокого машинного обучения. Программы с использованием TensorFlow можно компилировать и запускать как на CPU, так и на GPU. Также данная библиотека имеет обширный встроенный функционал логирования, собственный интерактивный визуализатор данных и логов~\cite{muller}.

\subsection{Keras}

Keras используется для быстрого прототипирования систем с использованием нейронных сетей и машинного обучения. Пакет представляет из себя высокоуровневый API, который работает поверх TensorFlow или Theano. Поддерживает как вычисления на CPU, так и на GPU

\subsection{Scikit-learn}

Scikit-learn -- это одна из самых популярных библиотек для языка Python, в которой реализованы основные алгоритмы машинного обучения, такие как классификация различных типов, регрессия и кластеризация данных. Библиотека распространяется свободно и является бесплатной для использования в своих проектах~\cite{rashka}.


Данная библиотека создана на основе двух других -- NumPy и SciPy, имеющих большое количество готовых реализаций часто используемых математических и статистических функций. Библиотека хорошо подходит для простых и средней сложности задач, а также для людей, которые только начинают свой путь в изучении машинного обучения.

\subsection{PyTorch}

PyTorch -- это популяный пакет Python для глубокого машинного обучения, который можно использовать для расширения функционала совместно с такими пакетами как NumPy, SciPy и Cython. Главной функцией PyTorch является возможность вычислений с использованием GPU. Отличается высокой скоростью работы и удобным API-интерфейсом расширения с помощью своей логики, написанной на C или C++.

\subsection{Theano}
Theano -- это библиотека, в которой содержится базовый набор инструментов для машинного обучения и конфигурирования нейросетей. Так же у данной библиотеки есть встроенные методы для эффективного вычисления математических выражений, содержащих многомерные массивы~\cite{rashka}.


Theano тесно интегрирована с библиотекой NumPy, что дает возможность просто и быстро производить вычисления. Главным преимуществом библиотеки является возможность использования GPU без изменения кода программы, что дает преимущество при выполнении ресуркоемких задач. Также возможно использование динамической генерации кода на языке программирования C~\cite{douson}.


\subsection{Обзор существующих решений реализации графического интерфейса для задач машинного обучения}

Прежде чем разрабатывать приложение из данной работы была произведена попытка найти существующие готовые решения для текущей задачи с графическим интерфейсом. Были найдены всего лишь два решения: проект на GitHub MachineLearningGUI и программный комплекс Weka. Рассмотрим каждое из них отдельно.


MachineLearningGUI -- десктопное приложение, написанное на языке программирования Python с помощью библиотеки PyQt. Работает только с библиотекой Scikit-learn и алгоритмом классификации с деревом принятия решений. Так же автор проекта указал в описании, что приложение работает только с одним набором данных, который идет вместе с проектом. Интерфейс программы состоит из четырех вкладок, каждая из которых отвечает за конкретный шаг: загрузка обучающей выборки, препроцессинг данных, запуск алгоритма и просмотр результатов. На рисунках~\ref{fig:github_gui_ml_1} и~\ref{fig:github_gui_ml_2} представлены первый и третий шаги.

\imgh{0.9}{github_gui_ml_1}{Интерфейс первого шага с загрузкой файла обучающей выборки}


\imgh{0.9}{github_gui_ml_2}{Интерфейс третьего шага с запуском алгоритма}


В целом проект выглядит сыро и не кажется пригодным для использования, по крайней мере для задачи из данной работы.


Weka -- открытый программный комплекс, содержащий в себе реализации алгоритмов машинного обучения для решения задач интеллектуального анализа. Проект разработан на языке программирования Java на базе университета Вайкато в Новой Зеландии. Целью проекта является создание современной среды для разработки и применения методов машинного обучения к реальным данным и упрощения этого процесса. Weka широко используется в учебных целях и исследователями в области машинного обучения. В состав комплекса входят средства для препроцессинга данных, классификации, регрессии, кластеризации и визуализации результатов\cite{weka1}.


Для работы в Weka необходимо загрузить файл с обучающей выборкой. На вкладке Preprocess (рисунок~\ref{fig:weka_1}) можно увидеть статистические метрики, расчитанные по выборки и применить один или несколько фильтров к набору данных.

\imgh{0.95}{weka_1}{Интерфейс программного комплекса Weka - вкладка Preprocess}


На вкладке Classify происходит выбор алгоритма классификации, выбор столбца для класса и запуск процесса классификации (рисунок~\ref{fig:weka_2}).


\imgh{0.95}{weka_2}{Интерфейс программного комплекса Weka - вкладка Classify}


На вкладке Visualize имеется возможность построить графики с распределением выборки (рисунок~\ref{fig:weka_3}).


\imgh{0.95}{weka_3}{Интерфейс программного комплекса Weka - вкладка Visualize}


В целом, Weka может помочь решить большинство задач машинного обучения за счет обширного функционала, но данная программа кажется слишком перегруженной для задачи из текущей работы. Интерфейс кажется перегруженным и нужно сидеть разбираться в нем.
